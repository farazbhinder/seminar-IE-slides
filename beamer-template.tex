\documentclass{beamer}
\usepackage[english]{babel}
\usepackage[utf8]{inputenc}

\usepackage{color}
\usepackage{graphicx}
\usepackage{fancybox}


% for tick marks
%\usepackage{dingbat}

\usepackage{beamerthemesplit}
\usetheme[compress]{MPIjast}


\title[Toponym Resolution and Adaptive Context Features]{Toponym Resolution and Adaptive Context Features}
\subtitle{}
\author[Faraz Ahmad]{Faraz Ahmad}
\date{\today}
\institute[MPI-Inf]{
Max Planck Institute for Informatics\\
Saarland Informatics Campus\\
\color{MPIIblue}{s8faahma@stud.uni-saarland.de}}

%---------------------------------------%
%---------- RECURRING OUTLINE ----------%
% have this if you'd like a recurring outline
\AtBeginSection[]  % "Beamer, do the following at the start of every section"
{
\begin{frame}<beamer> 
\frametitle{Outline} % make a frame titled "Outline"
\tableofcontents[currentsection,hideallsubsections]  % show TOC and highlight current section
\end{frame}
}
%----------------------------------------


\begin{document}
\frame[plain]{\titlepage}
\frame{\frametitle{Outline}\tableofcontents[hideallsubsections]}

%========================================
%========================================

\section[Introduction]{Introduction}

%###############################################

%###############################################

%----------------------------------------


\subsection{Terminology}
%
%\frame{
%\frametitle{Terminology}
%
%\begin{itemize}
%\item Toponyms
%\item Geotagging
%\begin{itemize}
%	\item Toponym recognition
%	\item Toponym disambiguation
%\end{itemize}
%\item Difficulty in Geotagging
%\item Gazeteers
%\end{itemize}
%} % END OF FRAME

%----------------------------------------

\subsection{Toponyms}
\frame{
\frametitle{Toponyms}
\begin{block}{Toponyms}
	Words in a document text that correspond to \textbf{location names} are called toponyms.
\end{block}

\begin{block}{refer to a \textbf{populated place}, such as: }
	\begin{itemize}
		\item[(1)] Cities/Towns
		\item[(2)] States/Provinces
		\item[(3)] Countries
	\end{itemize}
\end{block}
} % END OF FRAME

%----------------------------------------

\frame[t]{
\frametitle{Toponyms - examples}

\begin{block}{Toponyms examples: }
	\begin{itemize}
		\item[(1)]  ``\underline{\textbf{France}} won the world cup in 1998."
		\pause
		\item[(2)] ``\underline{\textbf{Birmingham}}, a major city of \underline{\textbf{West Midlands}} is the second most populous city of \underline{\textbf{England}}, third most populous is \underline{\textbf{Leeds}}."
		\pause
	\end{itemize}
\end{block}

\begin{columns}
	\begin{column}{.5\textwidth}
		\begin{figure}
			\includegraphics[width=\textwidth]{france.png} 
			%\caption{AvH in Wikipedia, \textit{Source: URL}}
		\end{figure}
	\end{column}
	\begin{column}{.5\textwidth}
		\begin{figure}
			\includegraphics[width=\textwidth]{uk.png} 
			%\caption{AvH in Wikipedia, \textit{Source: URL}}
		\end{figure}
	\end{column}
	
\end{columns}

\vfill
} % END OF FRAME

%========================================

\subsection{Geotagging}


\frame{
\frametitle{Geotagging}

\begin{block}{Geotagging}
	understanding the {geographic content} of a text document.
\end{block}

\begin{block}{is a two step process: }
	\begin{itemize}
		\item[(1)] detecting the toponyms
		\item[(2)] grounding the detected toponyms
	\end{itemize}
\end{block}

\vfill
} % END OF FRAME

%----------------------------------------

\frame{
\frametitle{Geotagging - example}
\begin{block}{sentence }
	``France won the world cup in 1998."
\end{block}
\pause
\begin{block}{step 1: detecting the toponyms}
	``\underline{France} won the world cup in 1998."
\end{block}
\pause
\begin{block}{step 2: grounding the detected toponyms}
``\underline{France} (lat 46, long 2) won the world cup in 1998."
\end{block}

\vfill
} % END OF FRAME

%----------------------------------------

\frame{
	\frametitle{Geotagging - the two steps defined}
	
	\begin{itemize}
		\item Toponym recognition (Geoparsing)
		\item Toponym resolution (Geocoding)
	\end{itemize}
	 
	 
	\begin{block}{Toponym recognition}
		\textbf{detect} all the \textbf{toponyms} (location
		names) in text.
	\end{block}
	\begin{block}{Toponym resolution}
		\textbf{assign} the \textbf{correct lat/long value} to \textbf{toponyms}.
	\end{block}
	
	\vfill
} % END OF FRAME

%----------------------------------------

\frame{
	\frametitle{Geotagging - example}
	\begin{block}{sentence }
		``France won the world cup in 1998."
	\end{block}
	\begin{block}{step 1: detecting the toponyms = Toponym recognition}
		``\underline{France} won the world cup in 1998."
	\end{block}
	\begin{block}{step 2: grounding the detected toponyms = Toponym resolution}
		``\underline{France} (lat 46, long 2) won the world cup in 1998."
	\end{block}
	
	\vfill
} % END OF FRAME

%========================================

\subsection{Difficulty in Geotagging}


\frame{
\frametitle{Difficulty in Geotagging}

\begin{block}{in toponym recognition}
	need to understand natural language in order to detect toponyms accurately.
\end{block}

\begin{block}{in toponym resolution}
	need to understand document's content to ground a toponym to its correct lat/long value \textbf{(correct interpretation)}.
\end{block}


} % END OF FRAME

\subsection{Difficulty in Geotagging - example}


\frame{
	\frametitle{Difficulty in Geotagging - ambiguity}
	
	\begin{itemize}
		\item geo/geo ambiguity
		\item geo/non-geo ambiguity
	\end{itemize}

	\pause
	
	\begin{block}{geo/geo ambiguity}
		\textbf{"Hyderabad"}, it is a city in Pakistan and India.
	\end{block}
	
	\pause
	
	\begin{block}{geo/non-geo ambiguity}
		\textbf{"Jordan"}, it is a name of a country and a person
	\end{block}
	
	
} % END OF FRAME
%========================================

\subsection{Gazeteers}


\frame{
	\frametitle{Gazeteers}
	
	\begin{block}{Gazeteers}
		Gazeteers are long lists of geographic locations with respective features such as lat/long values and usually type of location (city, state, country etc).
	\end{block}

	\begin{block}{some examples of online available gazeteers}
		\begin{itemize}
			\item GeoNames
			\item World Factbook by CIA
			\item Getty Thesaurus of Geographic Names
		\end{itemize}	
	\end{block}
	
} % END OF FRAME

%----------------------------------------

\frame{
	\frametitle{Gazeteers}
	
	\begin{block}{useful for}
		\begin{itemize}
			\item Toponym recognition (looking up)
			\item getting features* of toponyms
		\end{itemize}
	\end{block}
	
	$\ast$ population, no. of interpretations, alternative names, location type, lat/long value of the toponym
	
	\begin{figure}
		\includegraphics[width=\textwidth]{geonames.png} 
		\caption{GeoNames gazeteer}
	\end{figure}
		
} % END OF FRAME

%========================================
%========================================

\section[Motivation]{Motivation}


\subsection{Geographic content in queries}

\frame{
	\frametitle{Geographic content in queries}
	
	\begin{itemize}
		\item 14.8\% geographic queries in 2001 Excite search engine log
		\item 13\% geographic queries (related to USA) in 3 months data of AOL search engine in 2006 
	\end{itemize}
	
	
} % END OF FRAME

%----------------------------------------


%========================================
%========================================



\subsection{NERD vs Geotagging}


\frame{
\frametitle{NERD vs Geotagging}

\begin{table}[bt]
	\begin{tabular}{|c|c|} \hline
		\textbf{NERD}      & \textbf{Geotagging} \\ 
		\hline
		\uncover<2->{named entities} & \uncover<2->{only toponyms} \\
		\uncover<3->{knowledge bases} & \uncover<3->{gazeteers} \\
		\uncover<4->{ground = link} & \uncover<4->{ground = lat/long} \\
		\uncover<5->{knowledgebases are general}  & \uncover<5->{gazeteers are specific} \\ 
		\hline
	\end{tabular}
\end{table}

%\begin{columns}
%	\begin{column}{.5\textwidth}
%			\alert{NERD}
%			\begin{itemize}
%				\item disambiguate all entities
%				\item uses knowledgebases to ground entities
%				\item ground = link to the correct knowledgebase entry about entity
%				\item knowledgebases are more general
%			\end{itemize}
%	\end{column}
%	\begin{column}{.5\textwidth}
%			\alert{Geotagging}
%			\begin{itemize}
%				\item disambiguate only toponyms
%				\item uses various features to ground toponyms
%				\item ground = assign correct lat/long value to the toponym
%				\item gazeteers are more geographically focussed
%				\item gazeteers have many toponyms
%			\end{itemize}
%	\end{column}
%\end{columns}

%\pause

%\begin{block}{Bottom line}
%	Geotagging deserves attention in its own right, and NERD cannot be used for it.
%\end{block}

\vfill
} % END OF FRAME

%----------------------------------------
\subsection{NERD for doing Geotagging?}


\frame{
	\frametitle{NERD for doing Geotagging?}
	
	\begin{overprint}
		\onslide<1>
		\begin{figure}
			\includegraphics[width=\textwidth]{aida.png} 
			\caption{from AIDA Web Interface, \textit{Source: URL}}
		\end{figure}
		
		\onslide<2>
		\begin{figure}
			\includegraphics[width=\textwidth]{aida1.png} 
			\caption{from AIDA Web Interface, \textit{Source: URL}}
		\end{figure}
	
		\onslide<3>
		\begin{figure}
			\includegraphics[width=\textwidth]{aida2.png} 
			\caption{from AIDA Web Interface, \textit{Source: URL}}
		\end{figure}
	
		\onslide<4>
		\begin{figure}
			\includegraphics[width=\textwidth]{ambi.png} 
			\caption{from Ambiverse Entity Linking Demo, \textit{Source: URL}}
		\end{figure}
	
	\end{overprint}
	
	\vfill
} % END OF FRAME

%----------------------------------------
%========================================


\section[Adaptive Context Features]{Adaptive Context Features}

\subsection{Toponym recognition procedure}
\frame{
	\frametitle{Toponym recognition procedure}
	
	\begin{itemize}
		\item tokenization
		\item lookups
		\item statistical NLP tools
		\item POS tagging
	\end{itemize}
} % END OF FRAME
%----------------------------------------

\subsection{Toponym resolution procedure}
\frame{
	\frametitle{Toponym resolution procedure}
	
	\begin{itemize}
		\item cast as classification problem
		\item identify each toponym/interpretation pair as correct or incorrect
		\item used decision trees
		\item can get confidence score for each decision
	\end{itemize}
\vfill
} % END OF FRAME
%----------------------------------------

\subsection{Input features }
\frame{
	\frametitle{Input features for resolution}
	
	\begin{itemize}
		\item Context free features
		\item Adaptive context features
	\end{itemize}

	\pause

	\begin{block}{Context free features}
		 do not depend on the context (window) of the toponym t being resolved; usually available from gazeteers
	\end{block}

	\pause
	
	\begin{block}{Adaptive context features}
		depend on the context (window) around the toponym t being resolved; other toponyms in the window	around toponym t help in resolving it
	\end{block}
\vfill
} % END OF FRAME
%----------------------------------------

\subsection{Context Free Features}
\frame{
	\frametitle{Context Free Features}
	
	\begin{columns}
		\begin{column}{.3\textwidth}
				\begin{itemize}
				\item Interpretations
				\item Population
				\item Altnames
				\item Dateline
				\item Locallex
			\end{itemize}
		\end{column}
		\begin{column}{.7\textwidth}
			\pause
			\begin{figure}
				\includegraphics[width=\textwidth]{dateline.png} 
				\caption{Dateline toponym example, \textit{Source: nytimes Feb 4, 2017.}}
			\end{figure}
		\end{column}
		
	\end{columns}
	
	\vfill
} % END OF FRAME
%----------------------------------------

\subsection{Adaptive Context Features}
\frame{
	\frametitle{Adaptive \textbf{Context} Features}
	
	\begin{block}{\textbf{Context} of a toponym t}
	the other toponyms in the window around toponym t
	\end{block}

	\begin{overprint}
		\onslide<1>
		\begin{figure}
			\includegraphics[width=\textwidth]{adaptive.png} 
			\caption{from Lieberman and Samet, \textit{Source: URL}}
		\end{figure}
					
	\end{overprint}

} % END OF FRAME
%----------------------------------------

\frame{
	\frametitle{\textbf{Adaptive} Context Features}
	
	\begin{block}{\textbf{Adaptive}}
		means that the window's breadth $w_b$ and depth $w_d$ can be changed
	\end{block}
	
	%\begin{block}{Depth $w_d$ of window}
	%	the number of interpretations to consider for each toponym in the window
	%\end{block}
	
	\begin{overprint}
		\onslide<1>
		\begin{figure}
			\includegraphics[width=\textwidth]{adaptive.png} 
			\caption{from Lieberman and Samet, \textit{Source: URL}}
		\end{figure}
		
		\onslide<2>
		\begin{figure}
			\includegraphics[width=\textwidth]{adaptive.png} 
			\caption{from Lieberman and Samet, \textit{Source: URL}}
		\end{figure}
		
		\vspace*{-5cm}
		\begin{block}{\centering advantage of adaptiveness}
			\centering {$w_b$ and $w_d$ can be changed to make the resolution process faster or more accurate!}
		\end{block}
		
	\end{overprint}
	
} % END OF FRAME
%----------------------------------------

\subsection{1. Proxmity Features}
\frame{
	\frametitle{1. Proximity Features}
	
	\begin{block}{1. Proximity Features}
		represent the average distance of an interpretation $l_t$ of toponym $t$, to the geographically closest interpretations, say $l_o$ of all the other toponyms $o$ in the window.
	\end{block}
		\begin{overprint}
		\onslide<1>
		\begin{figure}
			\includegraphics[width=\textwidth]{adaptive-proximity-0.png} 
			\caption{from Lieberman and Samet, \textit{Source: URL}}
		\end{figure}
		
		
	\end{overprint}
} % END OF FRAME
%----------------------------------------

\subsection{Computing Proximity Features}
\frame{
	\frametitle{Computing Proximity Features}
	
	\begin{overprint}
		\onslide<1>
		\begin{figure}
			\includegraphics[width=\textwidth]{adaptive-proximity.png} 
			\caption{from Lieberman and Samet, \textit{Source: URL}}
		\end{figure}
		
		\onslide<2>
		\begin{figure}
			\includegraphics[width=\textwidth]{adaptive-proximity-a.png} 
			\caption{from Lieberman and Samet, \textit{Source: URL}}
		\end{figure}
		
		\onslide<3>
		\begin{figure}
			\includegraphics[width=\textwidth]{adaptive-proximity-b.png} 
			\caption{from Lieberman and Samet, \textit{Source: URL}}
		\end{figure}
		
		\onslide<4>
		\begin{figure}
			\includegraphics[width=\textwidth]{adaptive-proximity-c.png} 
			\caption{from Lieberman and Samet, \textit{Source: URL}}
		\end{figure}
		
		\onslide<5>
		\begin{figure}
			\includegraphics[width=\textwidth]{adaptive-proximity-d.png} 
			\caption{from Lieberman and Samet, \textit{Source: URL}}
		\end{figure}
		
		\onslide<6>		
		\begin{figure}
			\includegraphics[width=\textwidth]{adaptive-proximity-e.png} 
			\caption{from Lieberman and Samet, \textit{Source: URL}}
		\end{figure}
		
		\onslide<7>
		\begin{figure}
			\includegraphics[width=\textwidth]{adaptive-proximity-f.png} 
			\caption{from Lieberman and Samet, \textit{Source: URL}}
		\end{figure}
		
		\onslide<8>
		\begin{figure}
			\includegraphics[width=\textwidth]{adaptive-proximity-g.png} 
			\caption{from Lieberman and Samet, \textit{Source: URL}}
		\end{figure}
		
		\onslide<9>
		\begin{figure}
			\includegraphics[width=\textwidth]{adaptive-proximity-h.png} 
			\caption{from Lieberman and Samet, \textit{Source: URL}}
		\end{figure}
		
		\onslide<10>
		\begin{figure}
			\includegraphics[width=\textwidth]{adaptive-proximity-i.png} 
			\caption{from Lieberman and Samet, \textit{Source: URL}}
		\end{figure}
		
		\onslide<11>
		\begin{figure}
			\includegraphics[width=\textwidth]{adaptive-proximity-j.png} 
			\caption{from Lieberman and Samet, \textit{Source: URL}}
		\end{figure}
		
		\onslide<12>
		\begin{figure}
			\includegraphics[width=\textwidth]{adaptive-proximity-j1.png} 
			\caption{from Lieberman and Samet, \textit{Source: URL}}
		\end{figure}
		
		\onslide<13>
		\begin{figure}
			\includegraphics[width=\textwidth]{adaptive-proximity-k.png} 
			\caption{from Lieberman and Samet, \textit{Source: URL}}
		\end{figure}
		
		\onslide<14>
		\begin{figure}
			\includegraphics[width=\textwidth]{adaptive-proximity-k1.png} 
			\caption{from Lieberman and Samet, \textit{Source: URL}}
		\end{figure}
		
		\onslide<15>
		\begin{figure}
			\includegraphics[width=\textwidth]{adaptive-proximity-k2.png} 
			\caption{from Lieberman and Samet, \textit{Source: URL}}
		\end{figure}
		
		\onslide<16>
		\begin{figure}
			\includegraphics[width=\textwidth]{adaptive-proximity-k3.png} 
			\caption{from Lieberman and Samet, \textit{Source: URL}}
		\end{figure}
		
		\onslide<17>
		\begin{figure}
			\includegraphics[width=\textwidth]{adaptive-proximity-k3.png} 
			\caption{from Lieberman and Samet, \textit{Source: URL}}
		\end{figure}
		
		\vspace*{-5cm}
		\begin{block}{\centering Proximity features}
			\centering  \textbf{the interpretation $l_t$, for which proximity score is lowest, is the chosen as the correct interpretation of $t$}
		\end{block}
		
	\end{overprint}
} % END OF FRAME
%----------------------------------------

\subsection{Proximity Features idea}
\frame{
	\frametitle{Proximity Features idea}
	
\begin{block}{\centering Proximity features}
	\centering  \textbf{all toponyms $o$ in the window around $t$ play a role in determining correct interpretation $l_t$ of $t$}
\end{block}

\pause

\begin{block}{\centering News articles}
	\centering  \textbf{generally talk about a specific region, so a news article will have many proximate toponyms}
\end{block}

	
} % END OF FRAME

%----------------------------------------

\frame{
	\frametitle{Proximity Features - example}
	
	``...,	Birmingham and Perth. It was relatively warm in London ...''
	
	\pause
	
	Assuming $w_d$=2, 
	
		\begin{tabular}{ c| c | c | c}
		{} & Birmingham & \textbf{Perth} & London \\
		\hline
		1 & UK & Scotland & UK \\
		2 & Michigan, USA & Australia & Ontario, Canada \\
		\end{tabular}
	
	\pause
	
		\begin{tabular}{c l l}
		$\checkmark$ & (Perth, Scotland) $\xrightarrow{distance}$ (Birmingham, UK) & 338.8 miles\\
		{} & (Perth, Scotland) $\xrightarrow{distance}$ (Birmingham, Michigan) & 3496.2 miles \\
		\pause
		$\checkmark$ & (Perth, Scotland) $\xrightarrow{distance}$ (London, UK) & 450.6 miles \\
		{} & (Perth, Scotland) $\xrightarrow{distance}$ (London, Ontario) & 3403.7 miles \\
		\end{tabular} \\
		
		... \\
		\pause
		prox score(Perth, Scotland) = average(338.8, 450.6) = 394.7\footnotemark miles \\
				
	\only<2->{\footnotetext[1]{The authors used geometric median for average}}
		
	\vfill
} % END OF FRAME

%----------------------------------------
\subsection{Proximity Features - example}
\frame{
	\frametitle{Proximity Features - example}
	
	``...,	Birmingham and Perth. It was relatively warm in London ...''
	
	\pause
	
	Assuming $w_d$=2, 
	
	\begin{tabular}{ c| c | c | c}
	{} & Birmingham & \textbf{Perth} & London \\
	\hline
	1 & UK & Scotland & UK \\
	2 & Michigan, USA & Australia & Ontario, Canada \\
	\end{tabular}
	
	\pause
	
	prox score(Perth, Scotland) = average(338.8, 450.6) = 394.7\footnotemark[1] miles  \\
	
	\pause
	
	similarly we calculate prox score for second $(t, l_t)$ pair  \\
	
	
	\begin{tabular}{c l l}
	$\checkmark$ & (Perth, Australia) $\xrightarrow{distance}$ (Birmingham, UK) & 9085 miles\\
	{} & (Perth, Australia) $\xrightarrow{distance}$ (Birmingham, Michigan) & 11,175 miles \\
	$\checkmark$ & (Perth, Australia) $\xrightarrow{distance}$ (London, UK) & 9007 miles \\
	{} & (Perth, Australia) $\xrightarrow{distance}$ (London, Ontario) & 11,245 miles \\
	\end{tabular} \\
	
	
	
	\pause
	
	prox score(Perth, Australia) = average(9085, 9007) = 9046\footnotemark[1] miles  \\
	
	\pause
	
	\begin{overprint}
		\vspace*{-3cm}
		\begin{block}{\centering Proximity features}
			\centering So the \textbf{correct interp.} for Perth, in this context, using Prox feature only would be \textbf{(Perth, Scotland)}! \\
			\centering as prox score(Perth, Scotland) \textless prox score(Perth, Australia)
		\end{block}
	\end{overprint}
	
	

	
	
	\only<2->{\footnotetext[1]{The authors used geometric median for average}}
	

	\vfill
} % END OF FRAME
%----------------------------------------


\subsection{2. Sibling Features}
\frame{
	\frametitle{2. Sibling Features}
	
	\begin{block}{2. Sibling Features}
		capture the toponyms that belong to same country, state	or any other administrative division
	\end{block}
	
	\pause
	
	\begin{itemize}
		\item capture sibling relations
		\item capture containment relationships 
	\end{itemize}
	
	\pause
	
	e.g. ``Saarbruecken and Voelklingen" are siblings at city level \\
	
	\pause
	
	e.g. "Paris, Texas" refers to Paris in Texas, USA \\
	
	\pause
	
	\textbf{computed in a similar way as the proximity features!}
	
} % END OF FRAME
%----------------------------------------

\subsection{Sibling Features idea}
\frame{
	\frametitle{Sibling Features idea}
	
	\begin{block}{\centering Sibling features}
		\centering  \textbf{play a role in determining correct interpretation for cities that are geographically far away but have common parent such as state or country}
	\end{block}
	
	\pause
	
	\begin{block}{\centering Sibling features}
		\centering  \textbf{can correctly capture relationships which may be considered geographically distant}
	\end{block}
	
	\vfill
} % END OF FRAME
%----------------------------------------

\subsection{Sibling Features - example}
\frame{
	\frametitle{Sibling Features - example}
	
	``... Bloomington, Rochester and Duluth. ..."
	
	\pause
	
	Assuming $w_d$=3, 
	
	\begin{tabular}{ c| c | c | c}
		
		{} & Bloomington & \textbf{Rochester} & Duluth \\
		\hline
		1 & Minnesota, USA & Victoria, Australia & Georgia, USA\\
		2 & Michigan, USA & New York, USA & Minnesota, USA\\
		3 & {} & Minnesota, USA & {}\\
	\end{tabular} \\
	
	... \\
	\pause
	
	
	\textbf{The sibling feature scores for toponym/interp pairs will be}
	\begin{tabular}{ c| c | c | c}
		
		{} & $(t, l_t)$ & Score & common sibling levels. \\
		\hline
		1 & (Rochester, Victoria Australia) & 0 & -\\
		\pause
		2 & (Rochester, New York USA) & 4 & 4 x USA\\
		\pause
		3 & (Rochester, Minnesota USA) & 6 & 4 x USA, 2 x Minnesota\\
	\end{tabular}
	
	\pause
	
	\begin{overprint}
		\vspace*{-2.5cm}
		\begin{block}{\centering Sibling features}
			\centering \textbf{correct interp.} = \textbf{(Rochester, Minenesota)}! \\
			\centering as sibling score(Rochester, Minnesota) is \textbf{largest} among all toponym/interp pairs!
		\end{block}
	\end{overprint}
	
	\vfill
} % END OF FRAME
%----------------------------------------

\subsection{interpretations order}
\frame{
	\frametitle{interpretations order}
	
	``... Bloomington, Rochester and Duluth. ..."
	
	\pause
	
	Assuming \textbf{$w_d$=2}, 
	
	\begin{tabular}{ c| c | c | c}
		
		{} & Bloomington & \textbf{Rochester} & Duluth \\
		\hline
		1 & Minnesota, USA & Victoria, Australia & Georgia, USA\\
		2 & Michigan, USA & New York, USA & Minnesota, USA\\
	\end{tabular} \\
	
	... \\
	\pause
	
	
	\textbf{The sibling feature scores for toponym/interp pairs will be}
	\begin{tabular}{ c| c | c | c}
		
		{} & $(t, l_t)$ & Score & common sibling levels. \\
		\hline
		1 & (Rochester, Victoria Australia) & 0 & -\\
		\pause
		2 & (Rochester, New York USA) & 4 & 4 x USA\\
	\end{tabular}

	\vfill
} % END OF FRAME
%----------------------------------------

\subsection{feature propagation}
\frame{
	\frametitle{feature propagation}
	
	\begin{itemize}
		\item toponym t considered once in window
		\item feature propagation - after toponym resolution
	\end{itemize}
	
	\vfill
} % END OF FRAME
%----------------------------------------
\subsection{Evaluation measures}
\frame{
	\frametitle{Evaluation measures}
	
	\begin{itemize}
		\item Precision
		\item Recall
		\item F1 score
	\end{itemize}
	
	\pause

	Precision = $\frac{\text{no. of correctly resolved toponyms}}{\text{no. of total resolved toponyms}}$ \\
	...\\
	Recall = $\frac{\text{no. of correctly resolved toponyms}}{\text{no. of true toponyms}}$

	\vfill
} % END OF FRAME
%----------------------------------------


\subsection{Evaluation results}
\frame{
	\frametitle{Evaluation results}
	
	\begin{itemize}
		\item  news data from ACE, LGL and CLUST datasets
		\item  compared with two commercial geotagging frameworks i.e. (Thomson Reuter's OpenCalais and Yahoo!'s Placemaker)
	\end{itemize}
	
	\pause
	
	\textbf{performed better in majority of test scenarios and datasets} \\
	
	(details in the original paper)
	
} % END OF FRAME
%----------------------------------------


%========================================
%========================================

\section[Applications]{Applications}

%\subsection{Highlighting}

%----------------------------------------

\def\hilite<#1>{%
	\temporal<#1>{\color{black}}{\color{MPIIred}}%
	{\color{gray}}}

%----------------------------------------

\subsection{Geographic visualization and browsing}
\frame{
	\frametitle{Geographic visualization and browsing}
	
	\begin{itemize}
		\item thematic maps
		\item geographic anchoring of encyclopedia articles
	\end{itemize}
		
} % END OF FRAME

%----------------------------------------

\subsection{Question answering}
\frame{
	\frametitle{Question answering}
	
	\textbf{trend in search engines to answer some questions directly, instead of returning search results} \\
	
	\pause
	
	e.g. consider the query ``What is the distance between between Lahore and Multan?" \\
	...\\
	
	\pause
	
	can be solved by
	\begin{itemize}
		\item (1) resolve both toponyms in the query, 
		\item (2) compute the geographic distance using a geometric formula
	\end{itemize}
	
} % END OF FRAME

%----------------------------------------

\subsection{Improving search results}
\frame{
	\frametitle{Improving search results - for geographic queries}
	
	\textbf{return relevant documents which would not be returned using keyword based search} \\
	...\\
	
	\pause
	
	e.g. query is "Saarland" \\
	
	should also return documents which mention cities of Saarland, but not the keyword \textbf{Saarland} in them! \\
	...\\
	\pause
	
	can be done by assigning \textbf{geographic focus} to documents
	
} % END OF FRAME


%----------------------------------------

\subsection{Browsing news geographically}
\frame{
	\frametitle{Browsing news geographically}
	
	
	e.g. system \textbf{NewsStand System: }	
	users can choose a region of interest and read news and articles relevant to it\\
	or browse news geographically on world map in  an interactive manner
	\pause
	\bigskip
	\begin{itemize}
		\item crawls around 50,000 news article every day
		\item cluster them on content and \textbf{location}
		\item associate each cluster to its \textbf{geographic focus}
		\item stories appear as the map as the map is zoomed in/out
	\end{itemize}
	
} % END OF FRAME

%----------------------------------------

\subsection{Browsing news geographically - NewsStand}
\frame{
	\frametitle{Browsing news geographically - NewsStand}
	
	
	\begin{overprint}
		\onslide<1>
		\begin{figure}
			\includegraphics[width=\textwidth]{aus0.png} 
			\caption{NewsStand System, \textit{Source: URL}}
		\end{figure}
		
		\onslide<2>
		\begin{figure}
			\includegraphics[width=\textwidth]{aus1.png} 
			\caption{NewsStand System, \textit{Source: URL}}
		\end{figure}
		
		\onslide<3>
		\begin{figure}
			\includegraphics[width=\textwidth]{aus2.png} 
			\caption{NewsStand System, \textit{Source: URL}}
		\end{figure}
	\end{overprint}
	
	\vfill
} % END OF FRAME
%========================================
%========================================



\section[Conclusion]{Conclusion}
\subsection{Conclusion}
\frame{
	\frametitle{Conclusion}
	
	\begin{itemize}
		\item Geotagging an important process which allows to
		\begin{itemize}
			\item find toponyms (Recognition)
			\item resolve/ground them (Resolution)
		\end{itemize}
		\item existing NER techniques can help in recognition
		\item many features can be used to help in resolution
		\item geotagging has many applications/usecases
	\end{itemize}
		
} % END OF FRAME


%----------------------------------------
\subsection{Questions}
\frame{\frametitle{Questions}

\begin{center}\begin{LARGE}\textbf{Questions?}\end{LARGE}\end{center}


}

\subsection{References}
\frame{
	\frametitle{References}

	\begin{itemize}
		\item  Michael D. Lieberman and Hanan Samet. Adaptive Context Features for Toponym Resolution in Streaming News. In In SIGIR'12: Proceedings of the 35th International ACM SIGIR Conference on Research and Development in Information Retrieval, pages 731$-$740, 2012.
		\item Jochen L Leidner. Toponym Resolution in Text : Annotation, Evaluation and Applications of Spatial Grounding of Place Names. Universal Press, Boca Raton, FL, USA, 2008.
		\item Michael D Lieberman, Hanan Samet, and Jagan Sankaranarayanan. Geotagging with local lexicons to build indexes for textually-specified
		spatial data. In 2010 IEEE 26th International Conference on Data Engineering (ICDE 2010), pages 201$-$212. IEEE, 2010.
	\end{itemize}

\vfill
}

\newcommand{\backupbegin}{
   \newcounter{framenumberappendix}
   \setcounter{framenumberappendix}{\value{framenumber}}
}
\newcommand{\backupend}{
   \addtocounter{framenumberappendix}{-\value{framenumber}}
   \addtocounter{framenumber}{\value{framenumberappendix}} 
}

\appendix
\backupbegin


\backupend

\end{document}
